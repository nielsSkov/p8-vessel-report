\chapter{Test Journal: GPS Variances Test} \label{app:GPSImprovement}

\textbf{Date: TBA}

\subsubsection*{Purpose}
Dertermine the effectiveness of including base corrections when determining GPS loaction
\subsection*{Equipment}
\begin{itemize}
	\item Emlid Reach RTK GPS connected to the base station as described in Appendix \ref{app:rtk_gps}
    \item Water proof container
    \item HTC Desire HD with access to cellular networks
\end{itemize}

\subsubsection*{Theory}
\cite{RTK_GPS}\cite{ambg_res}\cite{GPS_errors}\cite{novatel}.

The inaccuracies of GPS receivers is mostly due to atmospheric changes, which adds a verity of disturbances dependent on the weather, temperature etc.
The most significant error to be corrected is the ambiguity resolution. 
Ambiguity resolution is a fixed offset the GPS receiver experiences as the number of wavelengths from the satellite to the receiver is unknown. 
This results in a integer which needs to be resolved, to improve the accuracy of the GPS measurements. 
If the ambiguity resolution is not found, the gained precision from other sources is mostly irrelevant due to the bias.\cite{novatel}\cite{RTK_GPS}\cite{ambg_res}

This results in most modern GPS receivers having a precision of 2-5 m. 
For applications where this precision is insufficient, an RTK GPS can be used. 
By comparing the measurements received from a base station, the rover is able to compensate for these disturbances.
The base station is a stationary GPS receiver with a known position, which streams correction data, to be used by the rover. 
The increased precision gained from including a base station is done by assuming the position errors are correlated.
As long as the rover remains within a close distance to the base station, this is a good assumption. 
The rover uses the correction data received form the base to estimate the noise causing the errors. 
This result in the position position being improved by orders of magnitude. 
%The increased performance is a result of the weather pattern is locally similar, meaning that if the base station and the rover is close (within 10-20 km) to each other, the disturbances they experience is the same.
The further the rover gets from the base, the more the noises start to become uncorrelated, resulting in a loss in accuracy.\cite{EmlidRTK}\cite{RTK_GPS}
%This enables the rover to compensate for the weather patterns, granted the base station received the same satellites as the rover.

By comparing the Mean and variance of similar GPS with and without base correction, the theoretical gain in accuracy by including base corrections can be shown.
Under optimal circumstances, this test should be preformed by two similar GPS's at the simultaneously, however it was not a possibility, thus two days with similar weather was chosen as test dates instead.

\subsection*{Procedure}

\begin{enumerate}
	\item Through reach view connect the Emlid Reach to the base station.
	\item Turn on Hotspot on the phone.
	\item Put the GPS and Phone in the container.
	\item Place the GPS and phone with access to a power outlet and clear sky view.
	\item Wait 24 hours.
	\item Retrieve the log process the data.
	\item Disconnect the phone and restart the experiment without base corrections.
	
\end{enumerate}

\subsubsection*{Results}
%\begin{figure}[H]
%    \includegraphics[width=.7\textwidth]{figures/GPSVariances}
%\end{figure}
%%
%\begin{flalign}
%    \sigma_{x\mathrm{n,GPS}}^2 & = 0.00050346 \ \mathrm{m}^2  \nonumber \\
%    \sigma_{y\mathrm{n,GPS}}^2 & = 0.00057036 \ \mathrm{m}^2  \nonumber
%\end{flalign}
