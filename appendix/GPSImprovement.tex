\chapter{Test Journal: GPS Variances Test} \label{app:GPSImprovement}

\textbf{Date: TBA}

\subsubsection*{Purpose}
Dertermine the effectiveness of including base corrections when determining GPS loaction
\subsection*{Equipment}
\begin{itemize}
	\item Emlid Reach RTK GPS connected to the base station as described in Appendix \ref{app:rtk_gps}
    \item Water proof container
    \item HTC Desire HD with access to cellular networks
\end{itemize}

\subsubsection*{Theory}

By comparing the Mean and variance of similar GPS with and without base correction, the theoretical gain in accuracy by including base corrections can be shown.
Under optimal circumstances, this test should be preformed by two similar GPS's at the simultaneously, however it was not a possibility, thus two days with similar weather was chosen as test dates instead.

\subsection*{Procedure}

\begin{enumerate}
	\item Through reach view connect the Emlid Reach to the base station.
	\item Turn on Hotspot on the phone.
	\item Put the GPS and Phone in the container.
	\item Place the GPS and phone with access to a power outlet and clear sky view.
	\item Wait 24 hours.
	\item Retrieve the log process the data.
	\item Disconnect the phone and restart the experiment without base corrections.
	
\end{enumerate}

\subsubsection*{Results}
%\begin{figure}[H]
%    \includegraphics[width=.7\textwidth]{figures/GPSVariances}
%\end{figure}
%%
%\begin{flalign}
%    \sigma_{x\mathrm{n,GPS}}^2 & = 0.00050346 \ \mathrm{m}^2  \nonumber \\
%    \sigma_{y\mathrm{n,GPS}}^2 & = 0.00057036 \ \mathrm{m}^2  \nonumber
%\end{flalign}
