\chapter{Test Journal: Added Mass}
%%%% Purpose
\subsection{Purpose}
The purpose of this test journal is to measure the added mass of the vessel, which is needed for completing the model.
%%%% Theory
\subsection{Theory}
By knowing the Force applied to the vessel and the acceleration, the added mass can be obtained by rearranging equation \autoref{eq:x_pos_model} and \autoref{eq:y_pos_model} to acquire the x and y components respectively.
This assumes the force-input relationship of the actuators as well as the drag coefficients is known.

%%%% Requirements
\subsection{Equipment}
\begin{itemize}
	\item Aau ship\fxnote{Ensure this is the correct name, is it enough to just mention a ship or every thing located on it as this will influence the test}
	\item Emlid Reach RTK GPS
	\item 2x INLINE 750 14.8V brushless DC-Motor
	\item 2x Speed Controller +70 G3.5
	\item ADIS16405BMLZ IMU
	\item Ps3 Controller  
	\item 2x Smarthpone with WiFi-Hotspot enabled
	\item laptop running remote-controller-client.by and it's dependencies installed (script located in appendix)\fxnote{Make sure this is done at hand in}
\end{itemize}


\subsection{Test Setup}



%%%% Test Setup
\subsection{Procedure}
\begin{enumerate}
	\item Turn on all equipment
	\item launch following launch scripts:
		\begin{itemize}
			\item addedMass.launch 
			\item Rosbag.launch \fxnote{inset correct script name}
		\end{itemize}
	\item Connect Ps3-controller to laptop
	\item Run remote-controller-client.py with sudo privileges
	\item input a step to the motors by holding down the 
	\item reposition the vessel to the initial starting position
	\item two different steps to the motors
\item process the data using \autoref{eq:x_pos_model} and \autoref{eq:y_pos_model}
\end{enumerate}


\subsection{Error Sources}
\begin{itemize}
	\item Wind and Wave Disturbances
	\item Measurement noise
\end{itemize}

%%%% Results
\subsection{result}
