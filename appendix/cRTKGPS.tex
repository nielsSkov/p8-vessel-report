\chapter{RTK GPS}\label{app:rtk_gps}
This appendix contains a description on how a RTK gps works, how the basestation is implemented and how to use it.


\section{RTK GPS basics}
%%http://www.novatel.com/an-introduction-to-gnss/chapter-5-resolving-errors/real-time-kinematic-rtk/
%%https://www.e-education.psu.edu/geog862/node/1828
%%http://www.wasoft.de/e/iagwg451/intro/introduction.html
The inaccuracies of GPS receivers is mostly due to atmospheric changes, which adds a verity of disturbances dependent on the weather, temperature etc.
This results in most modern GPS receivers having a precision of 2-5 m. 
For applications where this precision is insufficient, an RTK GPS can be used. 
By comparing the measurements received from a base station, the rover is able to compensate for the disturbances.
The base station is a stationary GPS receiver with a known position, which streams correction data, to be used by the rover. 
The increased performance is a result of the weather pattern is locally similar, meaning that if the base station and the rover is close (within 10-20 km) to each other, the disturbances they experience is the same.
This enables the rover to compensate for the weather patterns, granted the base station received the same satellites as the rover.
.....[good intro] can be broken into 3 steps, namely ambiguity fixing, correction coefficient estimation,
This can result in down to sub centimeter precision for some high grade implementations.

\section{Base station implementation and usage}
The modules used for the base station is the emlid reach RTK.
The emlid reach is setup to send the RTCM3 messages through a USB interface to a computer, acting as a bridge between the university servers and the outside. 
RTCM3 is a protocol designed to send correction data from a base station to a rover. 


It consists of different message types, varying in length.
Each message contains a message header and a checksum. 
The header is initiated with a hex value of d3, indicating that a new package begins. 
As the data is transmitted binary, this pattern is not a guarantee that a message is transmitted. 
To ensure the correct package is send, the checksum can be used to check if the package is in the right format. 
The message header contains information on message type and packet size, which is used to forward the right amount of bytes each time.

The computer runs a python server capable of reading the serial data, and forwards in the correct package sizes.
The package size is obtained by locating the initialization sequence to find the header containing the package size.
The checksum can be implemented to ensure the package is read correctly, however through initial testing, it was found that this was sufficient to check packet sizes instead to get a reliable package transfer, however this feature might improve reliability in the future.
The base station is implemented to cast the RTCM3 message on a TCP socket on IP: 192.38.55.85 Port: 5500. 
Users with access to the Internet is able to access this socket from anywhere.

