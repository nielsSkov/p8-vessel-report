Autonomous surface vehicles (ASV) have a wide range of applications that goes from marine research to surveillance. The aim of this project is to design a control strategy for an autonomous surface vessel with the purpose of acquiring bathymetric measurements. The vessel at hand is equipped with two thrusters, a Real Time Kinematic (RTK) GPS module, an Inertial Measurement Unit (IMU) and two processing units. The control system is divided in two levels, an inner and an outer controller. The former is designed using both a Linear Quadratic Regulator (LQR) and robust control theory, and handles the heading and speed of the vessel. The performance and the robustness of these two approaches is also compared and analyzed. The outer controller, on the other hand, makes the vessel track a path by computing its appropriate heading and speed using a enclosure based steering algorithm. This control system receives data from the vessel sensors through two Kalman filters, which estimate the attitude and translational variables of the vessel.