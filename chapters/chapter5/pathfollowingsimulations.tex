\section{Simulation Results}
The path following algorithm has been tested in the same path and considering different settings for the algorithm. In all cases, the radius of the circle defined around the vessel is 5 m and the distance in which the active waypoints are changed is 3 m. \fxnote{check these numbers}

In \autoref{fig:lqrwrong}, \ref{fig:distlqrwrong}, \ref{fig:robwrong} and \ref{fig:distrobwrong} the results of the algorithm are presented when considering the simpler case in which $\psi_\mathrm{ref} = \chi$, that is, assuming the velocity of the vessel is pointing along the $x_\mathrm{b}$ direction. The simulations are performed with both inner controller designs. In these graphs, it is clear that the path is not precisely followed when disturbances are introduced in the system. This offset is expected as the assumption for this simpler algorithm to work does not hold with disturbances like wind and waves.
\begin{figure}[H]
	\captionbox  %<--use captionbox instead if no global caption is needed
	{               %                                \%-%-%-%-%-%-%\
		Performance of the path following algorithm based on $\psi_\mathrm{ref}=\chi$ and using the LQR inner controller. The system is experiencing wind and wave disturbances, model perturbations and measurement noise.\label{fig:lqrwrong}                                  %\
	}                                                                 %\
	{                                                                  %\
		\includegraphics[width=.45\textwidth]{figures/path_lqr_no_correc}         %\
	}                                                                    %\
	\hspace{5pt}                                                          %\
	\captionbox  %<-----------------------------------------------------%\
	{       
		Distance to the path when using the algorithm based on $\psi_\mathrm{ref}=\chi$ and the LQR inner controller .The system is experiencing wind and wave disturbances, model perturbations and measurement noise.                                       %\                         %\
		\label{fig:distlqrwrong}                                     %\
	}                                                                           %\
	{                                                                            %\
		\includegraphics[width=.45\textwidth]{figures/dist_lqr_no_correc}            %|
	}                                                                             %|
\end{figure}
\begin{figure}[H]
	\captionbox 
	{   
		Performance of the path following algorithm based on $\psi_\mathrm{ref}=\chi$ and using the $\mathcal{H}_\infty$ inner controller. The system is experiencing wind and wave disturbances, model perturbations and measurement noise.\label{fig:robwrong}
	}                                                                 
	{                                                                  
		\includegraphics[width=.45\textwidth]{figures/path_rob_no_correc}         
	}                                                                    
	\hspace{5pt}                                                          
	\captionbox  
	{      
		Distance to the path when using the algorithm based on $\psi_\mathrm{ref}=\chi$ and the $\mathcal{H}_\infty$ inner controller .The system is experiencing wind and wave disturbances, model perturbations and measurement noise.\label{fig:distrobwrong}
	}                                                                          
	{
		\includegraphics[width=.45\textwidth]{figures/dist_rob_no_correc}
	}
\end{figure}
When the information of the vessel velocity is used to calculate the reference angle, $\psi_\mathrm{ref}$, the disturbance is rejected. This is seen in \autoref{fig:lrqcorrect}, \ref{fig:distlqr}, \ref{fig:robustcorrect} and \ref{fig:distrobustcorrect}, where the vessel is experiencing disturbances and model perturbations in the same range as in the previouly shown figures. In this case, the offset in position has been corrected and the path is followed within the desired precision in the straight line segments. There are still some small deviations due to the wave disturbance present in the system. 
\begin{figure}[H]
	\captionbox  %<--use captionbox instead if no global caption is needed
	{               %                                \%-%-%-%-%-%-%\
		Performance of the path following algorithm based on $\psi_\mathrm{ref}=\chi-\beta$ and using the LQR inner controller. The system is experiencing wind and wave disturbances, model perturbations and measurement noise.                %\
		\label{fig:lrqcorrect}                                  %\
	}                                                                 %\
	{                                                                  %\
		\includegraphics[width=.45\textwidth]{figures/path_lqr}         %\
	}                                                                    %\
	\hspace{5pt}                                                          %\
	\captionbox  %<-----------------------------------------------------%\
	{       
			Distance to the path when using the algorithm based on $\psi_\mathrm{ref}=\chi-\beta$ and the LQR inner controller .The system is experiencing wind and wave disturbances, model perturbations and measurement noise.                                                                  %\                         %\
		\label{fig:distlqr}                                     %\
	}                                                                           %\
	{                                                                            %\
		\includegraphics[width=.45\textwidth]{figures/dist_lqr}            %|
	}                                                                             %|
\end{figure}
\begin{figure}[H]
	\captionbox 
	{   
		Performance of the path following algorithm based on $\psi_\mathrm{ref}=\chi-\beta$ and using the $\mathcal{H}_\infty$ inner controller. The system is experiencing wind and wave disturbances, model perturbations and measurement noise. \label{fig:robustcorrect}
	}                                                                 
	{                                                                  
		\includegraphics[width=.45\textwidth]{figures/path_rob}         
	}                                                                    
	\hspace{5pt}                                                          
	\captionbox  
	{      
			Distance to the path when using the algorithm based on $\psi_\mathrm{ref}=\chi-\beta$ and the $\mathcal{H}_\infty$ inner controller .The system is experiencing wind and wave disturbances, model perturbations and measurement noise.\label{fig:distrobustcorrect}
	}                                                                          
	{
		\includegraphics[width=.45\textwidth]{figures/dist_rob}
	}
\end{figure}
According to the results of the simulations, it can be said that the vessel hits the waypoints when they are part of a straight line section of the path. In curved sections, the vessel joins smoothly the straight line segments that approximate the curve. In many cases, and especially for bathymetric measurements, the algorithm can be considered suitable.