\chapter{Conclusion}\label{chap:conclusion}

In this project, a control strategy for an ASV is developed to navigate through a given area for the purpose of taking bathymetric measurements.

The first step has been to analyze the problem and which requirements are needed for the system. These include requirements for the robustness and precision of the controller as well as for the final implementation. The analysis also contains a dynamic model of the system to be used in the control design, which describes the behavior of the vessel.

Then, the control strategy has been divided into two cascaded controllers. 

The inner controller is in charge of controlling the velocity and heading of the vessel, and it has been designed using and comparing two different approaches. The first approach has been an LQR, which is based on optimizing a cost function that includes the inputs and the convergence of the states. The second approach has been done using $\mathcal{H}_\infty$ theory, and it has been used to design a controller robust against disturbances and noise. Both controllers have been simulated and compared to analyze their performance.

In the area to survey, the path is generated by calculating the waypoints needed to cover it. Then, the outer controller calculates, using an enclosure based steering algorithm, the heading and speed references to send to the inner controller, such that the vessel follows the path. As in the case of the inner controller, its performance has also been analyzed though simulations that include disturbances, noise and varying parameters.

Finally, an estimator has been designed to fuse the data from both the IMU and the GPS. It consists of two Kalman filters, one for attitude and one for position, that estimate the needed variables for the controller to work such as heading, speed and position. The estimator has then been tuned and tested through simulation to check its performance.

Even though the simulated results have not been fully replicated in the real vessel, they show a promising behavior of the control system.

