\chapter{Conclusion}\label{chap:conclusion}

The aim of this project has been to design a control strategy for an autonomous surface vessel so it is able to navigate through a given area to take bathymetric measurements in water.

The first step has been to analyze the problem and which requirements are needed for the system. These include requirements for the robustness and precision of the controller as well as for the final implementation. The analysis also contains a dynamic model of the system to be used in the control design, which describes the behavior of the vessel both rotational and translational.

The control strategy has been divided into two cascaded controllers. 

The inner one is in charge of controlling the velocity and heading of the vessel, and it has been designed using and comparing two different approaches. The first approach has been a Linear Quadratic Regulator, which is based on optimizing a cost function that includes the inputs and the convergence of the states. The second approach has been done using $\mathcal{H}_\infty$ theory, and it is used to design a robust controller against disturbances and noise. Both controller has then been tested and compared in simulation to analyze their performance.

The outer controller has been designed to send reference commands to the inner controller for the vessel to follow the path. This path has been generated given the area to survey and calculating te needed waypoints to cover it. Then the outer controller calculates, using a enclosure based steering, the needed heading to follow the path and sends this command to the inner controller. As in the case of the inner controller, its performance has also been tested though simulation that includes disturbances, noise and varying parameters.

Finally, an estimator has been designed to fused the data from both the IMU and the GPS. It consists on two Kalman filters, one for attitude and one for position, that estimate the needed variables for the controller to work such as heading, speed or position. The estimator has then been tuned and tested through simulation to check its performance.

\fxnote{Talk about final statements or results}