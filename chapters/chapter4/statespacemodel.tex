\section{State Space Model}\label{chap:control}

The linearized model derived in \autoref{sec:linearizationModel}, consisting of \autoref{eq:x_pos_model_lin} to \ref{eq:psi_model_lin}, needs to be represented in state space form in order to design a state space controller. In order to do that, the 3 degree of freedom model used for the control of the vessel is represented as
\begin{flalign}
  \vec{\dot{x}}(t) &= \vec{A} \vec{x}(t) + \vec{B} \vec{u}(t)
  \label{xDotLinear} \\
  \vec{y}(t) &= \vec{C} \vec{x}(t) + \vec{D} \vec{u}(t)
  \label{yLinear} 
\end{flalign}
\begin{where}
  \va{\vec{x}}{is the state vector}{}
  \va{\vec{u}}{is the input vector}{}
  \va{\vec{y}}{is the output vector}{}
  \va{\vec{A}}{is the state matrix}{}
  \va{\vec{B}}{is the input matrix}{}
  \va{\vec{C}}{is the output matrix}{}
  \va{\vec{D}}{is the feed-forward matrix}{}
\end{where}
%
The state vector is constituted by the angle and velocity in yaw as well as the velocity in x in the body frame. The outputs of the system are yaw angle and velocity in x in the body frame. The input to the system is composed of the two forces applied in the body frame.
%It is important to notice that the position of the vessel in the body reference frame represents the  integration of the velocity along the body frame directions.
%This can also be seen as the position of the vessel with respect to a frame whose origin coincides with that of the NED frame and whose orientation coincides with that of the body frame \cite[p. 173]{TFossen}. 

\begin{minipage}{0.32\linewidth}
  \begin{flalign}
    \vec{x(t)} = 
    \begin{bmatrix}
      \psi\\
      \dot{\psi}\\
      \dot{x}_{b} \\
    \end{bmatrix} \nonumber
    \label{xVector}
  \end{flalign}  
\end{minipage}\hfill
%\hspace{0.03\linewidth}
\begin{minipage}{0.32\linewidth}
  \begin{flalign}
    \vec{y(t)} = 
    \begin{bmatrix}
      \phi \\
      \dot{x}_{b} \\
    \end{bmatrix} \nonumber
    \label{yVector}
  \end{flalign}
\end{minipage}\hfill
%\hspace{0.03\linewidth}
\begin{minipage}{0.32\linewidth}
  \begin{flalign}
    \vec{u(t)}= 
    \begin{bmatrix}
      F_1 \\
      F_2 
    \end{bmatrix}
    \label{uVector}
  \end{flalign} \nonumber
\end{minipage}\hfill

The resulting $\vec{A}$, $\vec{B}$, $\vec{C}$ and $\vec{D}$ matrices are
\begin{flalign}
  \vec{A} &=
  \begin{bmatrix}
    \ 0 & 1                   & 0                \ \ \ \\ 
    \ 0 & -\frac{d_\psi}{I_z} & 0                \ \ \ \\ 
    \ 0 & 0                   & -\frac{d_x}{m_x} \ \ \     
  \end{bmatrix}\rule{30px}{0px}
    \vec{B} = 
  \begin{bmatrix}
    \ 0               & 0                \ \ \ \\
    \ \frac{l_1}{I_z} & -\frac{l_2}{I_z} \ \ \ \\   
    \ \frac{1}{m_x}   & \frac{1}{m_x}    \ \ \
  \end{bmatrix}\rule{30px}{0px}
  \vec{C} =   
  \begin{bmatrix}
    \ 1 & 0 & 0  \ \ \ \\ 
    \ 0 & 0 & 1  \ \ \    
  \end{bmatrix}
   \label{eqStateSpaceABC}
\end{flalign}
and the $\vec{D}$ matrix is zero.
