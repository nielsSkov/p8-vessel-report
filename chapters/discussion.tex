\chapter{Discussion}\label{chap:discussion}
The results obtained shows that the controller requirements and the implementation requirements were sufficiently fulfilled, despite there being minor issues. Some requirements, such as \textbf{E}, were not fulfilled, but the concepts were proven.

The initial simulations indicated that the controllers performed satisfactory with regards to their respective design parameters. The LQR was shown to have a faster step response than the $\mathcal{H}_\infty$, while the $\mathcal{H}_\infty$ controller proved to be more robust towards noise and disturbances. From this it is further emphasized that the LQR is an optimal controller while the $\mathcal{H}_\infty$ controller is robust. It should be noted that the $\mathcal{H}_\infty$ design technique is known for producing a conservative controller with regards to performance.

The implementation of the controllers on the ASV was problematic. Both controllers in the real system were too aggressive and the vessel did not behave as simulated. From this it was determined that some dynamics of the system were not modeled. Even though the model was verified in \autoref{sec:modelVerification}, upon further inspection it was noticed that only a constant force was applied by the motors, thus neglecting the transients of the motors. During testing a significant delay was noticed in the response time of the motors. It should also be noted that the motors have a dead zone in the lower range of the operation area. This meant it was difficult for the motors to perform minor corrections, which degraded the performance of the controller. Due to these reasonings it was decided that the system model lacks a more detailed description of the motors.

During testing it was also seen that vessel was sensitive to wave disturbances. Since the vessel has a narrow hull, it is difficult for the vessel to reject these disturbances when these affect the $y_\mathrm{b}$ axis. This effect was visible in Klingenberg pond where the waters can be considered calm. This is concerning as the intended use is to survey Port of Aalborg, which is assumed to have larger disturbances. Once again, it should be reiterated that the simulations proved the designed controller is capable of rejecting these larger disturbances.

The ROS implementation was deemed satisfactory as it was possible to successfully simulate the controller as described in \autoref{app:ModelNode}. Unfortunately, the Kalman filter used to estimate position did not work as intended and was not tested in ROS when simulating the controllers. It did not work due to an error in the implementation and was not corrected because of time restrictions. To overcome this problem, the implementation from \cite{thesis} was used. Besides this minor error, the simulated Kalman filters estimate the attitude and position of the vessel when the measurements are subjected to the noise similar to the real system.

The RTK GPS implemented shows a 95 percentile which gives a precision well below 30 cm, when it receives base correction data. However, the base correction data in \autoref{app:GPSImprovement} indicates an accuracy error. This error may be due to reflections from surrounding buildings as well as the antenna on the vessel being highly directional. While the exact position of the ASV may not be accurate, the RTK GPS consistently gives precise data. This indicates that a THU under 30 cm is achievable given a correct base setup.

% However, the controller design proved to be an issue when testing the system in real life. 
% This indicates that there is some dynamics in the system that is not taken into account. 
% The lack of a good motor description in the system model is suspected to introduce a significant delay in the system, which could influence the dynamics of the vessel . 
% Additionally, the motors had a dead zone in the lower range of the operation area. 
% This means the motors have a hard time preforming minor corrections, as these tend to require smaller motor values, thus limiting the precision of the corrections the vessel is able to preform. 