\section{Topics}
Each topic is predefined to contain specific data that need to be exchanged between the nodes. This data can be anything from sensor data to commands for the motors.

The \lstinline[style=cinline]{/samples} topic contains a message that comes from the LLI.

The data decoded by the \lstinline[style=cinline]{/sensor_node} from the previous message, is published in the \lstinline[style=cinline]{/imu} topic, such as accelerometer, magnetometer and gyroscope data.

The \lstinline[style=cinline]{/gps_pos} topic includes the latitude and longitude of the vessel, as well as its relative position to the chosen origin of the NED frame.

Each Kalman filter node, \lstinline[style=cinline]{/KF_attitude_node} and \lstinline[style=cinline]{/KF_position_node}, publishes the estimated states in two topics, \lstinline[style=cinline]{/kf_attitude} and \lstinline[style=cinline]{/kf_position}. These states include angular and translational positions, velocities and accelerations.

The \lstinline[style=cinline]{/path_follower_node} publish the references for speed and heading in the \lstinline[style=cinline]{/control_reference} topic, to whom the \lstinline[style=cinline]{/controller_node} is subscribed.

Finally, the \lstinline[style=cinline]{/controller_node} publishes the commands, in the form of PWM, that the motor controllers need to apply in the \lstinline[style=cinline]{/lli_input} topic.

Also, the \lstinline[style=cinline]{/keyboard_teleop_node} is used to manually control the USV. 


\textit{In Part II, the required elements to construct the control system for an ASV have been designed. These started with an inner controller design based on two different control theories, namely, LQR and $\mathcal{H}_\infty$. Then, the outer controller design has been presented as a path following algorithm called enclosure based steering. The part also includes, in the form of two Kalman filters, the sensor data processing performed in order to obtain the necessary states for the different controllers. Finally, the implementation of the designs in ROS has been described by presenting the nodes and topics involved}