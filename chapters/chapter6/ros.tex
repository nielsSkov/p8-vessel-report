\section{ROS}
ROS is designed to be a communication infrastructure for a project, which allows multiple programs to communicate between each other. 
Each program runs in individual threads, called nodes, such that their timing is independent from each other, except for hardware limitations.  
This allows each node to run in parallel in multiple threads while still being able to share data between them.\\
The data sharing is done through topics, onto which the nodes can publish and subscribe to. 
The topic contains the data stored in a prespecified data structure, specified as a message, such that each node publishes data of the same type. 
\subsection{System overview}
The system consistes of two subsystems, a Low-Level-Interface (LLI) and a High-Level-Interface (HLI). 
The LLI is a Harware interface, responsible for reading all the sensors and sending control signals to the motors.
The HLI contains the sensor fusion, and the controllers, as well as a interface to the operator. 


\subsection{Route following node}
The route following uses waypoints together with position data to generate a reference for the low level controller. 
The method described in section\fxnote{Ref to section describing Route following node} describes the line between two waypoints as an affine linear line.  
This description have the issue of having singularity when describing vertical lines, making the slope infinite.  
To circumvent this, the implementation uses a different approach. 
While the general concept remains the same, the method for describing a line is different. 

