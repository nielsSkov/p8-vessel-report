\section{Linearization of Model equations}
To simplify the model, the nonlinear terms have been linearized. 
\eqref{eq:forcez} is the only translatoric force with unlinear terms. 
Using the first order taylor approximation around 0 which leads to:
\begin{flalign}
	F_{zb} = -\rho g A_{wg} \tilde{z_{n}}
\label{eq:forcezlinear}
\end{flalign}

The angular terms described in \eqref{eq:torqphi} and \eqref{eq:torqtheta} both have nonlinear terms with needs to be linearized. 
Similar to their translatoric counterparts, the equations have been linearaized using a first order Taylor approximation
\begin{flalign}
	T_{\phi} = -\rho g V \overline{GM_{L}}\cdot \tilde{\phi}
\label{eq:torquephilinar}
\end{flalign}
\begin{flalign}
	T_{\theta} = -\rho g V \overline{GM_{L}}\cdot \tilde{\theta}
\label{eq:torquethetalinar}
\end{flalign}
The first order Taylor approximation is only applicable in areas close to the operating point, however as the vessel should not deviate much from this during operation. 
The only value with potential to deviate a lot is $\theta$,which always is used relative to the boat frame, such that the dynamics prevents the orientation of the boat to deviate much between samples.



%\begin{flalign}
%    m_\mathrm{x} \ddot{x}_\mathrm{b} &=  F_\mathrm{1} + F_\mathrm{2}  - d_\mathrm{x} \dot{x}_\mathrm{b} + m_\mathrm{y} \dot{y_\mathrm{b}} \dot{\psi} - m_\mathrm{z} \dot{z}_\mathrm{b} \dot{\theta} - F_\mathrm{x}
%    \label{eq:x_pos_model} \\
%    m_\mathrm{y} \ddot{y_\mathrm{b}} &=  -d_\mathrm{y}\dot{y_\mathrm{b}}-m_\mathrm{x}\dot{x_\mathrm{b}}\dot{\psi}+m_\mathrm{z}\dot{z_\mathrm{b}}\dot{\phi}-F_\mathrm{y}
%    \label{eq:y_pos_model} \\
%    m_\mathrm{z} \ddot{z_\mathrm{b}} &=  -d_\mathrm{z}\dot{z_\mathrm{b}}+ m_\mathrm{b}\dot{x_\mathrm{b}}\dot{\theta}-m_\mathrm{y}\dot{y_\mathrm{b}} \dot{\phi}-F_\mathrm{z} \label{eq:z_pos_model}
%\end{flalign}
%
%\begin{flalign}
%    I_\mathrm{x}\ddot{\phi} &= -d_\mathrm{\phi} \dot{\phi}-(m_\mathrm{z}-m_\mathrm{y}) \dot{z}_\mathrm{b} \dot{y}_\mathrm{b}-(I_\mathrm{z}-I_\mathrm{y}) \dot{\theta } \dot{\psi}+T_\mathrm{\phi}  
%    \label{eq:x_inert_model} \\
%    I_\mathrm{y}\ddot{\theta} &= -d_\mathrm{\theta} \dot{\theta}-(m_\mathrm{z}-m_\mathrm{x}) \dot{x}_\mathrm{b} \dot{y}_\mathrm{b}-(I_\mathrm{x}-I_\mathrm{z}) \dot{\phi} \dot{\psi}+T_\mathrm{\theta}  
%    \label{eq:y_inert_model} \\
%    I_\mathrm{z}\ddot{\psi} &= -d_\mathrm{\psi} \dot{\psi}-(m_\mathrm{y}-m_\mathrm{x}) \dot{x}_\mathrm{b}\dot{y}_\mathrm{b}-(I_\mathrm{z}-I_\mathrm{y})\dot{\psi}\dot{\theta}+F_\mathrm{1}l_\mathrm{1}+F_\mathrm{2}l_\mathrm{2}+T_\mathrm{\psi} \label{eq:z_inert_model}
%\end{flalign}
